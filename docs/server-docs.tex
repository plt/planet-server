\documentclass{article}

\title{PLaneT Server Deployment Documentation}
\author{Jacob Matthews}

\begin{document}
\maketitle

This document explains the live enviroment in which the PLaneT server
runs. It explains some of how the PLaneT server code works, but its
main focus is to explain how the deployment works.

\section{The setup}

The machine that runs the planet server is a dedicated computer
running in the University of Chicago server room. It is known by two
hostnames: coach.cs.uchicago.edu and planet.plt-scheme.org. Basic
server administration for non-planet-related tasks is handled by
University of Chicago CS Tech Staff (so they handle the machine's
physical housing, hardware issues, and things like managing the
firewall and keeping the kernel up to date if necessary). They are
aware of the machine's role and they are in contact occasionally as
the need arises. They have root access to the machine; PLT Scheme
staff do not.

PLaneT-specific administration must be handled by us. While the
machine has an NFS mount and you can access normal CS home directories
from there, currently all PLaneT-specific configuration lives in
subdirectories off of /local (see below for details of what lives where).

\subsection{Core PLaneT Services}

The planet server's job is to run three services:
\begin{itemize}
\item Apache
\item the PLT Scheme Web Server
\item PostgreSQL
\end{itemize}

All three of these services must be running for the system to
function. Their roles are as follows:

\subsubsection{Apache}

Apache's job is to serve static content (other than PLaneT packages
themselves), rewrite URLs using \texttt{mod\_rewrite}, and to proxy incoming
requests for dynamic content to the PLT Scheme web server, which is
only accessible to the local host. We use Apache as the main web
server rather than exposing the PLT Scheme web server directly so that
we can take advantage of \texttt{mod\_rewrite} (though more recent versions of
the PLT Scheme web server offer more flexible mappings between URLs
and servlets, so this may not be necessary anymore) and to allow us to
run other services such as bug tracking software that are written
assuming an Apache frontend in the future.

We run Apache Web Server version 2.0.54 as user wwwplanet listening
on port 8080 of all interfaces. The firewall, administered by Tech
Staff, forwards requests on port 80 to requests on port 8080. (The
point of this scheme is to avoid the need for Apache to run as a
privileged user.)  Its htdocs directory (where Apache-served content,
including static HTML files and potentially PHP or other non-Scheme
pages lives) is

  \texttt{/local/webroot/htdocs}

Imporant subdirectories of htdocs are as follows:

\begin{itemize}
\item \texttt{300/}, \texttt{4.x/}:  \\
each of these directories does nothing but hold that repository's RSS
  feed, which is automatically maintained by the PLaneT server code
  running on the PLT Scheme web server.

\item \texttt{autoinstallers/}: \\
a packages directory hierarchy (about which more below), each file in
  the leaves of these directories is a .plt file that, when installed
  using the DrScheme .plt file installer, just requires (and thus
  installs) the corresponding planet package.
\item \texttt{css/}: \\
all planet css files
\item {images/:} \\
all planet images
\item {package-source/}: \\
a parallel package directory hierarchy that contains package sources
pre-rendered as HTML at the leaves.
\end{itemize}

The apache installation itself (i.e., the files for the binaries and
so on) lives in \texttt{/local/apache}.

The most important directory there is \texttt{conf}, which contains
Apache's configuration files. The planet-specific rules, including
proxying and rewrite rules, live in
   \texttt{httpd.conf}
in that directory.

\subsection{The PLT Scheme Web Server}

The PLT Scheme web server's job is to serve dynamic content, including
the two main user-interface servlets display.ss and add.ss which
manage user interaction with planet, and planet-servlet.ss, which
handles requests for planet packages from the PLaneT client.

The PLT Web Server runs as user wwwplanet listening on port 8081, is
blocked by firewall from the outside world and is only accessible from
the local machine, which means that its only inbound traffic is from
the Apache proxy. It is currently running mzscheme version v3.99.0.25
and a variety of PLaneT packages.

Imporant directories for this service:

\begin{itemize}
\item \texttt{/local/svn/plt/}: \\
Root of the SVN checkout that houses the version of PLT Scheme running
the planet server.

\item \texttt{/local/svn/iplt/}: \\
Root of the iplt checkout for planet, which houses two important
  subdirectories: \texttt{web}, which is the root of the standard PLT web
  templates (maintained by Eli Barzilay) and planet, where all the
  planet server code lives. The \texttt{iplt/planet} directory is used directly
  as the PLT server's servlets directory, so updates into that
  repository should be treated as live (it generally requires
  restarting the server or refreshing servlets to observe changes, but
  it is unwise to rely on this behavior for staging).
\item \texttt{/local/planet/archives/}: \\
This is the original package director hierarchy, though at this point
  the fact that it was the first one might only be of historical
  interest. The .plt file containing planet package "pkg.plt" owned by
  "owner" with package version x.y lives in the subdirectory
  \texttt{owner/pkg.plt/x/y/pkg.plt} off of this directory. The package's
  contents are also unpacked; they live in \texttt{owner/pkg.plt/x/y/contents/}
  (and whatever subdirectories may be present in the package itself).
\item \texttt{/local/planet/logs/}: \\
the file "logfile" in this directory contains the output written by
  the web server. This is generally where internal error messages
  get written.
\item \texttt{/local/planet/20x/}: \\
This is where very old packages live. The PLaneT server for the
  199-20x series of PLT Scheme releases had a different policy for
  repositories worked than the current policy. As a result when the
  planet system was rewritten in 2006 it was easier to effectively
  proxy all requests coming from v200 series planet clients to an old
  version of the planet server than it was to fit packages from
  version 200 into the new system. This directory holds those old
  packages. I doubt highly that anyone ever accesses them anymore, but
  it doesn't hurt to keep them around.
\item \texttt{/local/webroot/}: \\
Serves as the PLT Server's webroot, so mime.types, access log, and all
  the other standard PLT Server configuration stuff lives here with
  the exception of servlets. The servlets directory is actually a
  symlink to \texttt{/local/svn/iplt/planet}, as discussed above.
\item \texttt{/local/webroot/wwwplanet-home/}: \\
Home directory for user wwwplanet. This is significant because the
  subdirectory .plt-scheme in this directory contains all the planet
  packages used by the PLaneT server itself in serving PLaneT. It is
  sometimes necessary to directly manipulate these files because if
  they are not working then generally the planet server won't be
  working either.
\end{itemize}

Note that as of this writing I've got a pretty gross thing going on in
this directory. The database driver I use, \texttt{schematics/spgsql.plt}, has
a bug that causes it to throw an exception when certain escape
sequences are returned from the database in response to a query. This
actually happens in the planet data, and the schematics people seem to
be too busy to release a package that fixes the bug, so I fixed it
myself and just directly replaced the relevant file in the
package. \textbf{THIS IS A TERRIBLE THING TO HAVE DONE}, mostly because it
means that whenever the planet client feels compelled to download a
new version of spgsql.plt package it'll get the old, broken version
rather than the fixed one. It would be a very good idea to set up a
development link for a local fork of the package, copy the contents
into that directory, and then rewrite the require line in
iplt/planet/db.ss to point to that local copy; I just haven't done
so.

\subsection{PostgreSQL}

The PLT Scheme web server stores persistent data in a locally-running
PostgreSQL database. That database holds all metainformation about
packages --- their version numbers, their owners, and so on --- but it
does not hold the data of the packages themselves. It runs version 8.2
on port 5432 (which is the default for postgresql) as user
postgres. 

Note that as of this writing, PostgreSQL version 8.3 is NOT compatible
with the Scheme driver we use, a mismatch that manifests itself as
errors that appear to be quite high-level failures (such as queries
not finding rows that should be there) rather than obviously
protocol-level errors.

Important directories: 
\begin{itemize}
\item \texttt{/local/pgsql/}: \\
Root of all database stuff.

\item \texttt{/local/pgsql/data/}: \\
All database data. No permissions from anyone
except user postgres. Two files of interest in here:
\begin{itemize}
  \item \texttt{pg\_hba.conf}, which controls access policies for
  the database. (You shouldn't ever need to mess with even that, but
  you might need to remember that it doesn't contain the default
  information if you're upgrading the database software or something
  like that.)
  \item \texttt{logs/logfile}, the database's log. 
\end{itemize}
\end{itemize}

\subsection{Looking at the database}

The sql client lives at \texttt{/local/pgsql/psql} --- you'll need to
say the magic words \texttt{export PGDATA=/local/pgsql/data/}
so that it can find the database, then run
  \texttt{/local/pgsql/bin/psql planet}
to make queries. 

\textbf{WARNING: YOU ARE ACCESSING LIVE DATA! IT IS POSSIBLE
TO SCREW THINGS UP BADLY WITH AN INCORRECT UPDATE, DELETE, OR INSERT
HERE! TRY IT ELSEWHERE FIRST!}

\subsection{Additional utility}

\texttt{/local/planet/bin} contains helpers that make it easier to bring up
and take down planet services as necessary.

\begin{itemize}
\item \texttt{apachectl} --- symlink to the real apachectl program. 
\item \texttt{wwwplanet\_ctl} --- controls the PLT Scheme server instance
\item \texttt{planetdb\_ctl} --- controls the PostgreSQL instance
\end{itemize}

To use apachectl, you need to first \texttt{su wwwplanet} to have the right
permissions. The others can be run from your normal account; they will
prompt for the relevant password when run. They all have the same
interface:

\textit{command} \texttt{start}  --- brings up the service.
\textit{command} \texttt{stop} --- brings down the service.
\textit{command} \texttt{restart} --- restarts the service.

\texttt{apachectl} and \texttt{planetdb\_ctl} both just expose
functionality provided by apache and postgresql, respectively, so they
both support more options ("man apachectl" and "man pg\_ctl"
respectively); \texttt{wwwplanet\_ctl} by contrast is just a little shell script
I wrote so it is relatively feature-poor.

\section{Pragmatics: Doing things with the server}

It is extremely useful to have a mock server running elsewhere for
testing that you keep as close to the live server as possible. I use
Parallels on my mac, on which I've got a Debian installation set up
exactly like the above description. I always run development on that
server. When I want to update the planet servlets, I develop on my own
machine in a locally checked-out copy of the iplt tree, svn commit
that work, svn update on the testing server, restart at least the PLT
server, and then test the changes. Generally speaking, I try not to
update the Apache, mzscheme, or Postgres installations when it isn't
necessary. This is mainly to avoid breakage and the tradeoff is that
when I do need to update mzscheme it's generally extremely painful;
future maintainers should do what they feel is right.

To update the database, look on the web for general instructions on
how to update the database. Basically, the idea is to run pg\_dumpall,
which dumps database information to a file, then bring down the
database server, update the installation, start up the new server, and
then run pg\_restore to re-import the data.

Things that go wrong:

\begin{itemize}
\item Watch for permissions errors. In particular, if some Scheme function
  that accesses the filesystem works when you run it, it might not
  continue to work when wwwplanet runs it. This will generally cause
  bad things to happen on the server.

\item Make sure that you have the same versions in deployment as in
  testing. You never know when a slight change will cause something to
  break. (I don't know about the PostgreSQL 8.2 vs 8.3 difference from
  studying manuals!)

\item Try to make changes as compartmentalized as possible. If you have to
  update everything on the whole server at once, there's much more
  opportunity for things to break.

\item When in doubt, try restarting the PLT Scheme web server.
\end{itemize}
\end{document}
